% !Mode:: "TeX:UTF-8"
%%==========================
%% chapter01.tex for SJTU Master Thesis
%% based on CASthesis
%% modified by wei.jianwen@gmail.com
%% version: 0.3a
%% Encoding: UTF-8
%% last update: Dec 5th, 2010
%%==================================================

%\bibliographystyle{sjtu2} %[此处用于每章都生产参考文献]
\chapter{相关工作}
\label{chap:related}
就像之前说过的那样,人体姿势预测是一个很难的问题,特别是在没有上下文约束信息的情况下。
有一些研究者做了一些在三维(3D)空间去解决人体姿势识别问题的工作\cite{burenius20133d,ics14cvpr}。
在工作\cite{burenius20133d}中,他们将二维(2D)空间图画式框架算法\cite{ps1,ps2}扩展到了三维(3D)空间,并且提出了一个新的框架来对姿势位置、相邻躯干角度等。
Shotton等人在工作\cite{shotton2013real}中,提出了一个实时的算法去估计三维(3D)空间的人体姿势,
这是可以用在工业界的算法,因为实时性可以保证我们做出产品级的姿势识别程序。

\section{人体姿势预测}
大部分的人体姿势识别的工作都是基于二维(2D)空间的图片为数据基础,在这些工作中,他们用带方向的模板方块去表示一个人体躯干,
各个方块之间是独立分开的,这可以提供一个比较简单的方式来描述人体躯干之间的结构。
显而易见,这种方式对于很多比较复杂的姿势是无法处理的\cite{nips06,cvpr10,Daniel},比如有些躯干是被遮挡的,没有直接显现在图片中。
对于这种复杂的情况,在论文\cite{deva11}中,一种更高级的方式被提出来描述带方向的人体躯干,他们用多个带方向的模板来描述每一个人体躯干,
每一个带方向的模板被设定有一个属性“类别”。有趣的是,这个新的模型可以通过调整弹簧结构中相邻节点的关系而模拟这种躯干被遮挡的情况。

还有一些工作希望通过一些相关技术来增强人体姿势预测的精确度,比如图像分割技术等。
在论文\cite{eccv10}中,一堆图像特征(比如边缘反馈和区域分割)被结合进来增强人体姿势预测的精度。
在论文\cite{songchun}中,图像背景被描述为遵循高斯分布,通过高斯模型去获得更多的图像信息。
在论文\cite{mixing}中,他们采用了一种两步近似算法来提高视频中人体躯干中下臂的预测精度,这种算法的输出结果要求每一种解都跟周边有比较高的对比度。

在特征上,传统方法一般都是采用方向梯度算子(HOG)这类形状特征来进行特征的设计,
但是一些反应外观的特征(比如颜色、纹理特征等)对于人体姿势预测也是很有帮助的\cite{bmvc09}。
一般而言,这些反应外观的特征都是描述衣服的特征,因为数据集中人一般都是穿衣服的,而这些外观特征恰好描述了衣服属性。
通过图\ref{fig:eg}易知,人体躯干和衣服属性之间有很强的联系。
一些衣服属性识别的工作\cite{clotheccv,clothliu,poselets,junchi},他们首先进行人体姿势预测的算法,然后利用人体姿势预测的结果来作为衣服属性识别算法的输入,
由于有了人体姿势预测的结果,所以可以很容易的提取出人体躯干对应的衣服属性,从而进行衣服属性分析识别。
但是很明显,这种方法严重依赖于人体姿势预测的结果。
还有一些方法\cite{cloth12,shen2014unified},他们观察到不仅人体姿势预测的结果可以帮助到衣服属性的识别,而且衣服属性的识别也可以帮助提高人体姿势预测的精度。
所以这些方法将人体姿势预测问题和衣服属性识别问题联合起来,将两者的约束关系利用进来,同时提高人体姿势预测和衣服属性识别的精度。
然而,人体姿势预测问题的标准数据集一般都是只有人体躯干的标签信息,没有提供额外的衣服属性标签信息,所以这些方法都需要进行大量的衣服属性标注工作,
这无疑是非常耗时的一项工作。在我们的工作中,我们会将衣服属性作为隐变量加入到人体姿势预测问题中,所以避免了大量的、耗时的标注工作。

\section{衣服属性分析}
衣服属性分析也是计算机视觉领域中一个极其重要的问题,而且衣服属性分析的结果可以直接应用到电商领域。
还有很多其它工作也将衣服属性分析结合进各自的模型中,来提高精度。
在论文\cite{clothrec}中,Liu等人希望针对特定的场景来推荐合适的衣服。为了解决这个问题,首先需要建立一个模型来刻画图像底层特征和衣服推荐场景的关系,
他们提出了一个算法框架,先识别出衣服属性,然后通过衣服属性的语义信息来推荐匹配的衣服。
在论文\cite{action}中,跟本文类似的衣服属性技术被用到动作识别当中。
但是,有个关键的区别:论文\cite{action}中的衣服属性是被用作一个中间过程的结果,然后更高层的任务来采用衣服属性的结果去完成。
而在我们的工作中,衣服属性是跟人体躯干结合在一起的,构建一个联合模型来求解我们的问题。
我们提出的模型采用重标记的方法来同时优化人体躯干变量和衣服属性变量。

\section{隐变量结构式学习}
% ICML09_LatentSSVM
很多前人的工作都致力于将隐变量加入到判别式模型中,来通过加入隐变量的方式提高模型的预测精度。
这种想法来自于计算机视觉领域,在物体识别领域,一个很自然的想法是将人体躯干或者物体当作隐变量去建模。
论文\cite{HCRF}提出了隐式随机条件场,它是一种针对结构式预测的包含隐变量的判别式概率模型,
并且展示了该算法在两个计算机视觉领域的任务。
在自然语言理解领域,也有一些工作\cite{dlglv}来应用这种包含隐变量的判别式概率模型,
例如用这种判别式的方式来加入隐式的标注信息,来进行PCFG的训练。
由于加入了隐变量,所以一般结构式学习就会变为非凸问题,
所以一般采用基于梯度的方法来优化这种问题的非凸似然函数。

我们工作中采用的Concave-Convex(凹-凸)过程\cite{CCP}是最小化非凸函数的一般解决方案,这跟一般的非凸函数优化问题是一个类问题。
最近几年,出现了不少的应用来采用这个算法去解决实际问题,
包括去训练非凸的SVM(支持向量机)和直推式SVM(支持向量机)\cite{tcs}。
论文\cite{Vishwanathan05kernelmethods}中提高的方法,用CCCP去处理SVM和Gaussian Processes(高斯过程)中的缺失数据。
而类似的一篇文章\cite{llsvm}是非概率的模型,并避免了分割函数的计算,这对于结构式预测是非常关键的。
近年来,CCCP也被用来解决结构式预测中损失边界比较紧密的非凸问题\cite{boundSE}。

在计算机视觉领域,近年来有很多工作在用max-margin(最大化间隔)的规则去训练隐式随机条件场模型\cite{muldpm,maxHCRF}。
但是这些工作都是专注在分类问题上,只能说是结构式学习的特例而已。
论文\cite{muldpm}提出了一个最大化间隔形式和算法来进行结构化结果预测,并且可以使用隐变量。
这也是隐变量结构式学习领域中最经典的一篇工作。


\section{本章小结}
本章介绍了三个方面的相关工作,包括人体姿势预测、衣服属性分析和隐变量结构式学习。
可以看出人体姿势预测问题和衣服属性分析问题有着极其密切的联系,
二者其一都对另一个问题有极其大的帮助,所以我们通过建立关于两个问题的联合特征来提高人体姿势预测的精度。
由于标准数据集对衣服属性标签信息的缺失,我们自然而然想到了把衣服属性作为隐变量,
利用隐变量结构式学习去解这个问题。
