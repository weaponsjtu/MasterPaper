% !Mode:: "TeX:UTF-8"
%%==========================
%% chapter01.tex for SJTU Master Thesis
%% based on CASthesis
%% modified by wei.jianwen@gmail.com
%% version: 0.3a
%% Encoding: UTF-8
%% last update: Dec 5th, 2010
%%==================================================

%\bibliographystyle{sjtu2} %[此处用于每章都生产参考文献]
\chapter{实验结果}
\label{chap:exp}
\section{数据集}
我们在两个数据集上进行了评测,一个是Buffy数据集,一个是DL(daily life)数据集。
Buffy数据集包含了748张从Buffy TV秀中截取下来的图片,每张图片包含了人体各个部位的位置标签。
这个数据集是人体姿势预测问题的标准数据集,很多已有的工作都是在这个上面进行的。
DL数据集包含了997张人体生活照片,是我们从Flickr图片社交分享网站上抓取下来的。
我们对他进行了手工标注,标注了人体各个部位的位置标签。
跟Buffy数据集相比,DL数据集拥有更丰富的衣服属性信息。
为了对我们求解的衣服属性值进行评测,我们手动为Buffy和DL标注了衣服属性标签信息。
为了训练和评测,我们对每个数据集进行了切分。
对于Buffy数据集,选取472张图片作为训练数据集,剩下的作为测试数据集。
对于DL数据集,我们选取200张图片作为训练数据集,剩下的作为测试数据集。

\begin{table}
\centering
\caption{在Buffy数据集上和前人工作的对比}
\begin{tabular}{|c|c|c|c|c|c|} \hline
    Method & Torse & Upper arms & Lower arms & Head & Total \\ \hline
Andriluka et al.~\cite{cvpr09} &  90.7 & 79.3 & 41.2 & 95.5 & 73.5 \\ \hline
Sapp et al.~\cite{eccv10} & \textbf{100} & 95.3 & 63.0 & 96.2 & 85.5 \\ \hline
Yang and Ramanan~\cite{deva11} & \textbf{100} & 96.6 & 70.9 & \textbf{99.6} & 89.1 \\ \hline
Our Approach & \textbf{100} & \textbf{97.1} & \textbf{78.4} & 99.1 & \textbf{91.6} \\ \hline
\end{tabular}
\label{tb:buffy}
\end{table}

\begin{table}
\centering
\caption{在DL数据集上和前人工作的对比}
\begin{tabular}{|c|c|c|c|c|c|} \hline
    Method & Torse & Upper arms & Lower arms & Head & Total \\ \hline
Andriluka et al.~\cite{cvpr09} &  97.0 & 91.7 & 84.5 & 94.0 & 90.6 \\ \hline
Sapp et al.~\cite{eccv10} & \textbf{100} & 88.5 & 78.0 & 87.6 & 86.8 \\ \hline
Yang and Ramanan~\cite{deva11} & 99.8 & 95.7 & 87.5 & 95.6 & 93.6 \\ \hline
Our Approach & \textbf{100} & \textbf{97.2} & \textbf{91.3} & \textbf{99.1} & \textbf{95.7} \\ \hline
\end{tabular}
\label{tb:dl}
\end{table}

\section{实验评测规则}
我们和已有的三份工作进行了对比,包括Andriluka等\cite{},Sapp等\cite{},Yang and Ramanan等\cite{}。
对于人体姿势预测的结果,我们采用标准的评测指标,即就是正确姿势的概率PCP,它代表了正确的姿势所占的比例。
对于衣服属性的聚类结果,我们采用聚类工作中的标准指标F1分值来进行评测。
我们用KMeans聚类的结果作为衣服属性的基准,对于已有的三份工作,我们用他们得到的姿势解来求得最优的衣服属性,然后进行Kmeans聚类。

\section{和前人工作的对比}

\begin{figure}[tbp]
\centering
\includegraphics[width=0.9\textwidth]{img/compare.pdf}
\caption{ \textbf{我们的方法和 Yang and Ramanan~\cite{deva11}的对比 }
第1和3个结果是Yang and Ramanan~\cite{deva11}的方法产生的,我们发现这在上臂和下臂上得到了错误的结果。
而我们的方法(第2和4个)得到了正确的结果。}
\label{fig:compare}
\end{figure}

\section{实验结果分析}
图6展示了我们算法得出的一些结果,可以定性的看出,我们的算法取得了不错的效果。
在表2和表3中我们分别展示了Buffy和DL数据集的PCP评测结果。
对于Buffy数据集,表2显示我们的方法超越了目前最好的方法Yang and Ramanan\cite{}。
众所周知,下臂的检测是最具有挑战性的,但是令人惊喜的是,我们的方法比目前最好的方法超过了7.5个百分点,这是一个很大的提升了,显示了我们将衣服属性考虑进去的优越性。
对于DL数据集来说,我们的方法超越了目前已有的所有方法。

我们的方法附加的产出是对具有相似衣服属性的图片进行了聚类,图5分别展示了Buffy和DL数据集上的结果。
在图5的上半部分中,我们展示了Buffy数据集的聚类效果,DL数据集的聚类效果在下方。
表4和表5显示了每一种方法的F1分值,可以看出,我们的方法极大的提高了聚类的准确性,比基准要高出很多。
这很大程度上是因为我们的模型训练策略和迭代式的优化算法。
注意到,“Kmeans+Groundtruth“为我们的模型训练提供了初始的衣服属性标签信息,这也验证了我们方法的有效性。

\begin{table}
\centering
\caption{在Buffy数据集上对衣服属性结果的F1分值}
\begin{tabular}{|c|c|c|c|c|} \hline
    HPE & Sleeve & Neckline & Pattern & Total \\ \hline
Andriluka et al.~\cite{cvpr09} + $K$-Means & 24.1 & 26.6 & 34.2 & 28.3  \\ \hline
Sapp et al.~\cite{eccv10} + $K$-Means & 22.9 & 27.9 & 40.5 & 30.4 \\ \hline
Yang and Ramanan~\cite{deva11} + $K$-Means & 38.3 & 25.7 & 22.6 & 28.9\\ \hline
Groundtruth + $K$-Means & 34.7 & 36.1 & 39.5 & 36.8\\ \hline
Our Approach & \textbf{55.6} & \textbf{68.8} & \textbf{80.8} & \textbf{68.4}  \\ \hline
\end{tabular}
\label{tb:f1_buffy}
\end{table}


\begin{table}
\centering
\caption{在DL数据集上对衣服属性结果的F1分值}
\begin{tabular}{|c|c|c|c|c|} \hline
    HPE & Sleeve & Neckline & Pattern & Total \\ \hline
Andriluka et al.~\cite{cvpr09} + $K$-Means & 27.5 & 31.7 & 27.6 & 28.9  \\ \hline
Sapp et al.~\cite{eccv10} + $K$-Means & 34.9 & 30.5 & 23.8 & 29.7 \\ \hline
Yang and Ramanan~\cite{deva11} + $K$-Means & 43.2 & 28.6 & 35.8 & 35.9 \\ \hline
Groundtruth  + $K$-Means & 31 & 29.8 & 26.1 & 28.9 \\ \hline
Our Approach & \textbf{57.2} & \textbf{60.3} & \textbf{74.7} & \textbf{64.1}  \\ \hline
\end{tabular}
\label{tb:f1_dl}
\end{table}


\begin{figure*}[tbp]
\centering
\includegraphics[width=\textwidth]{img/attr.pdf}
\caption{ \textbf{Buffy上袖子的聚类结果和DL上衣领的聚类结果}
上面的框中的第一行表示袖子类别中的无袖类型,而第二行表示长袖类别。
下面的框中的第一行表示衣领属性中的尖领类别,而第二行代表圆领类别。
这两个框中的第二列都代表每一个属性的错误聚类结果。
}
\label{fig:sleeve}
\end{figure*}

\section{复杂度分析}

\begin{figure}
\centering
\includegraphics[width=\textwidth]{img/result.pdf}
\caption{ \textbf{我们的算法在Buffy和DL数据集上得到的结果示例}
上面两个框中的结果是Buffy数据集上的,剩下的是DL数据集上的。
我们用带方向的矩形框来表示人体姿势识别的结果。
每个数据集对应的第一个框是正确的结果,而第二个框是错误的结果。
红色的矩形框代表错误的结果,其它颜色分别代表不同的人体躯干部位。
}
\label{fig:result}
\end{figure}

\section{本章小结}
