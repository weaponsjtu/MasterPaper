% !Mode:: "TeX:UTF-8"
%%==================================================
%% abstract.tex for SJTU Master Thesis
%% based on CASthesis
%% modified by wei.jianwen@gmail.com
%% version: 0.3a
%% Encoding: UTF-8
%% last update: Dec 5th, 2010
%%==================================================

\begin{abstract}
近年来在图像分析、动作识别等领域,人体姿势预测这个基本问题得到了科学家们广泛的关注.
从已有的工作来看,人的头部、身躯等部位已经取得了很高的精度,但是手臂由于其丰富的姿势变化,目前还只有0.7的精度,这几乎是人体姿势预测领域最大的挑战。
一个明显的想法是,衣服属性信息对于姿势的精准预测是有很大的帮助.
前人也有很多利用衣服属性信息去帮助姿势预测的工作,但是需要人为的去标注大量的衣服属性标签,这是一项极其耗时的工作。

在这篇论文中,我们提出了一个将衣服属性作为隐变量的方法,采用隐变量结构式支持向量机算法去提高人体姿势预测的精度。
在我们提出的方法中,我们将衣服属性作为隐变量,这样就不需要显示的对衣服属性进行标注.
在此基础上,我们设计了良好的联合特征,并采用迭代式的结构式学习来进行模型的参数学习。
最后我们提出了一个近似迭代的推理算法来解决我们的有环图问题。
我们在两个真实的公开数据集上做了大量的实验,结果显示我们的算法取得了目前最好的精度,特别是在手臂的预测上,我们比前人工作提高了8个百分点。

\keywords{\large 姿势预测 \quad 结构式学习 \quad 隐变量}
\end{abstract}

\begin{englishabstract}
As a fundamental technique with wide applications in image parsing, action recognition, etc., 
human pose estimation (HPE) has been extensively investigated in recent years. 
For accurate and reliable estimation of the human pose, it is well-recognized that the clothing attributes are useful and should be utilized properly. 
Most previous approaches, however, require to manually annotate the clothing attributes and are therefore time-consuming. 
In this paper, by utilizing the latent structured support vector machines (LSSVM), we devise a \emph{latent} clothing attribute approach for HPE. 
Our approach models the clothing attributes as latent variables and thus requires no explicit labeling for the clothing attributes. 
We employ an interactive strategy to train the LSSVM model: in each iteration, one kind of variables (e.g., human pose or clothing attribute) are fixed and the others are optimized. 
Our extensive experiments on two real-world benchmarks show state-of-the-art performance of our proposed approach.

\englishkeywords{\large pose estimation, structure learning, latent variables}
\end{englishabstract}
