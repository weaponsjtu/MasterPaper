% !Mode:: "TeX:UTF-8"
%%==========================
%% chapter01.tex for SJTU Master Thesis
%% based on CASthesis
%% modified by wei.jianwen@gmail.com
%% version: 0.3a
%% Encoding: UTF-8
%% last update: Dec 5th, 2010
%%==================================================

%\bibliographystyle{sjtu2} %[此处用于每章都生产参考文献]
\chapter{总结与展望}
在这篇文章中,我们基于衣服属性和人体姿势之间紧密的联系,提出了将衣服属性作为隐变量,建立人体姿势和衣服属性的联合模型,从而提高人体姿势预测问题的精确度。
相比于前人的一些工作,我们提出的方法不需要进行额外的标注工作,节省了大量的时间,并且可以应用到工业界的大数据中。
对于这个结构化问题,我们采用隐变量结构式支持向量机框架来进行模型参数的学习。
首先,我们采用K-Means聚类算法对隐变量进行初始化,然后采用迭代的方式进行模型中人体姿势和衣服属性变量的更新。
由于我们加入了衣服属性变量,原来的图画式无环图问题变成了一个有环图问题,这对于测试图片的最优解预测有着极大的挑战。
我们提出了一个近似迭代算法,用因子图表示问题的结构,采用固定一类变量,然后更新另一类变量的方式求得问题的最优解。
最后,基于两个真实数据集,我们设计了大量实验来验证我们提出的方法。
实验结果表明我们的方法超越了目前最好的结果,并且在时间复杂度上大大降低,高效地解决了人体姿势预测问题。

当然我们的工作中还是有很多可以改进的地方,这也是我们以后的工作方向。
总体来说有以下几个方面:
(1)\textbf{衣服属性的自动挖掘。}衣服属性的定义和选择,是提高人体姿势预测很关键的一个环节。我们目前的思路是选择每个人体躯干对应的衣服属性,
只选择了几种衣服属性,这样肯定是不足的,还有很多很多的衣服属性可以帮助到人体姿势预测,理想的情况是应用到所有的这些信息。
我们希望在以后的工作中,可以提出一种自动挖掘衣服属性的算法,将所有可能帮助到人体姿势预测的衣服属性全部加入到我们的框架当中。
(2)\textbf{特征的自动化学习。}目前我们工作中是人工定义了衣服属性对应的底层图像特征,选择了跟对应的衣服属性最匹配的底层图像特征,
但是也有可能还会有更好的特征或者更好的特征组合。在以后的工作中,我们可以提出一种自动学习出最佳特征或者最佳特征组合的算法,
来为每一个衣服属性选择最佳特征,可能一些Feature Selection(特征选择)的方法有助于这个算法的形成。


